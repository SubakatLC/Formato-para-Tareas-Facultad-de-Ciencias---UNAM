%%%%%%%%%%%%%%%%%%%%%%%%%%%%
% PAQUETES
%%%%%%%%%%%%%%%%%%%%%%%%%%%%

% ==============================================================================
% CONFIGURACIÓN BÁSICA DEL DOCUMENTO
% ==============================================================================
\usepackage[utf8]{inputenc}                % Codificación UTF-8 para caracteres especiales
\usepackage[spanish]{babel}                % Localización en español (hyphenation, textos)
\usepackage{graphicx}                      % Manejo avanzado de imágenes [\includegraphics]
\usepackage{float}                         % Posicionamiento preciso de figuras/tablas [H]
\usepackage{bookmark}                      % Bookmarks interactivos en PDF (mejorado)
\usepackage[makeroom]{cancel}              % Cancelación de términos en ecuaciones [\cancel]

% ==============================================================================
% CONFIGURACIÓN DE HIPERVÍNCULOS Y REFERENCIAS
% ==============================================================================
\usepackage{hyperref}                      % Hipervínculos interactivos
\usepackage{nameref}                       % Referencias por nombre de sección
\usepackage{theoremref}                    % Referencias a teoremas mejoradas
\hypersetup{
    pdftitle={Assignment},                 % Metadatos del PDF
    colorlinks=true,                       % Enlaces en color (no recuadros)
    linkcolor=cyan!90,                     % Color enlaces internos
    urlcolor=blue,                         % Color URLs externas
    bookmarksnumbered=true,                % Numeración automática de bookmarks
    bookmarksopen=true                     % Bookmarks expandidos por defecto
}

% ==============================================================================
% PAQUETES MATEMÁTICOS Y SÍMBOLOS
% ==============================================================================
\usepackage{amsmath, amsfonts, amsthm, amssymb, mathtools} % Base para matemáticas
\usepackage{xfrac}                         % Fracciones diagonales [\sfrac]
\usepackage{tikz-cd}                       % Diagramas conmutativos
\usepackage{tikzsymbols}                   % Símbolos especiales (emojis, misc.)
\usepackage{siunitx}                       % Alineación numérica en tablas [S]
\usepackage{multicol}                      % Columnas múltiples en texto/matemáticas

% ==============================================================================
% FORMATO DE TABLAS Y ALGORITMOS
% ==============================================================================
\usepackage{booktabs}                      % Reglas profesionales para tablas [\toprule]
\usepackage{array}                         % Control avanzado de columnas en tablas
\usepackage[ruled,vlined,linesnumbered]{algorithm2e} % Escritura de algoritmos

% ==============================================================================
% LISTADOS DE CÓDIGO
% ==============================================================================
\usepackage{listings}                      % Inclusión de código fuente
\lstset{
    language=C,                            % Lenguaje base para resaltado
    basicstyle=\scriptsize\ttfamily,       % Estilo compacto tipo terminal
    numbers=left,                          % Numeración de línea izquierda
    numberstyle=\scriptsize\color{gray},   % Estilo numérico discreto
    frame=single,                          % Marco alrededor del código
    rulecolor=\color{black},               % Color del marco
    tabsize=2,                             % Tamaño de tabulaciones
    breaklines=true,                       % Ajuste automático de líneas
    keywordstyle=\color{blue},             % Palabras clave en azul
    commentstyle=\color{dkgreen},          % Comentarios en verde oscuro
    stringstyle=\color{mauve},             % Cadenas de texto en morado
    xleftmargin=4.0ex,                     % Sangría izquierda
    morekeywords={for,each,between,can,reach,in,is,Sort,Print,From} % Keywords adicionales
}

% ==============================================================================
% ELEMENTOS GRÁFICOS Y DE DISEÑO
% ==============================================================================
\usepackage[most,many,breakable]{tcolorbox} % Cajas decoradas con estilo
\usepackage{xcolor}                         % Gestión avanzada de colores
\usepackage{varwidth}                       % Cajas de ancho variable
\usepackage{enumitem}                       % Personalización de listas [nosep]

% ==============================================================================
% HERRAMIENTAS AVANZADAS Y MISCELÁNEOS
% ==============================================================================
\usepackage{etoolbox}                      % Utilidades para programación LaTeX
\usepackage{import}                        % Importación modular de archivos
\usepackage{xifthen}                       % Condicionales en documentos
\usepackage{pdfpages}                     % Inclusión de páginas PDF externas
\usepackage{transparent}                  % Manejo de transparencias
\usepackage[fixlanguage]{babelbib}        % Bibliografía en español
\bibliographystyle{babunsrt}             % Estilo de bibliografía
\usepackage{titlesec}                     % Personalización de títulos
\usepackage{comment}  


%%%%%%%%%%%%%%%%%%%%%%%%%%%%
% COLORES
%%%%%%%%%%%%%%%%%%%%%%%%%%%%

\definecolor{myg}{RGB}{56, 140, 70}
\definecolor{myb}{RGB}{45, 111, 177}
\definecolor{myr}{RGB}{199, 68, 64}
\definecolor{mytheorembg}{HTML}{F2F2F9}
\definecolor{mytheoremfr}{HTML}{00007B}
\definecolor{mylenmabg}{HTML}{FFFAF8}
\definecolor{mylenmafr}{HTML}{983b0f}
\definecolor{mypropbg}{HTML}{f2fbfc}
\definecolor{mypropfr}{HTML}{191971}
\definecolor{myexamplebg}{HTML}{F2FBF8}
\definecolor{myexamplefr}{HTML}{88D6D1}
\definecolor{myexampleti}{HTML}{2A7F7F}
\definecolor{mydefinitbg}{HTML}{E5E5FF}
\definecolor{mydefinitfr}{HTML}{3F3FA3}
\definecolor{notesgreen}{RGB}{0,162,0}
\definecolor{myp}{RGB}{197, 92, 212}
\definecolor{mygr}{HTML}{2C3338}
\definecolor{myred}{RGB}{127,0,0}
\definecolor{myyellow}{RGB}{169,121,69}
\definecolor{myexercisebg}{HTML}{F2FBF8}
\definecolor{myexercisefg}{HTML}{88D6D1}
\definecolor{mauve}{rgb}{0.58,0,0.82}  % Color morado
\definecolor{azulpuma}{RGB}{0,61,121} % Color azul puma
\definecolor{doradopuma}{RGB}{213,159,15} % Color dorado puma


%%%%%%%%%%%%%%%%%%%%%%%%%%%%
% CAJAS
%%%%%%%%%%%%%%%%%%%%%%%%%%%%
% ==============================================================================
% ENTORNOS PERSONALIZADOS PARA CAJAS DECORADAS
% ==============================================================================

% --------------------------
% Caja de Introducción/Indicaciones
% --------------------------
\newtcolorbox{intro}[1][]{%
    enhanced,                               % Habilitar mejoras gráficas
    before skip=2mm, after skip=2mm,        % Espaciado vertical
    colback=gray!5,                         % Color de fondo claro
    colframe=doradopuma!80!doradopuma,                % Borde oscuro
    boxrule=0.5mm,                          % Grosor del borde
    attach boxed title to top left={         % Posición del título
        xshift=1cm,                         % Desplazamiento horizontal
        yshift*=1mm-\tcboxedtitleheight      % Ajuste vertical preciso
    },
    varwidth boxed title*=-3cm,             % Ancho adaptable del título
    boxed title style={                     % Estilo personalizado del título
        frame code={                        % Diseño del marco del título
            \path[fill=tcbcolback]
            ([yshift=-1mm,xshift=-1mm]frame.north west)
            arc[start angle=0,end angle=180,radius=1mm]
            ([yshift=-1mm,xshift=1mm]frame.north east)
            arc[start angle=180,end angle=0,radius=1mm];
            \path[left color=doradopuma!60!black,
                  right color=doradopuma!60!black,
                  middle color=doradopuma!80!black]
            ([xshift=-2mm]frame.north west) -- ([xshift=2mm]frame.north east)
            [rounded corners=1mm]-- ([xshift=1mm,yshift=-1mm]frame.north east)
            -- (frame.south east) -- (frame.south west)
            -- ([xshift=-1mm,yshift=-1mm]frame.north west)
            [sharp corners]-- cycle;
        }, 
        interior engine=empty,              % Sin relleno adicional
    },
    fonttitle=\bfseries,                    % Título en negrita
    title={Indicaciones},                   % Texto predeterminado del título
    #1                                      % Permite personalización adicional
}

% --------------------------
% Caja de Bibliografía (similar a intro con título diferente)
% --------------------------
\newtcolorbox{biblio}[1][]{enhanced,
    before skip=2mm, after skip=2mm, colback=gray!5, colframe=black!80!black, boxrule=0.5mm,
    attach boxed title to top left={xshift=1cm, yshift*=1mm-\tcboxedtitleheight}, varwidth boxed title*=-3cm,
    boxed title style={frame code={
            \path[fill=tcbcolback]
            ([yshift=-1mm,xshift=-1mm]frame.north west)
            arc[start angle=0,end angle=180,radius=1mm]
            ([yshift=-1mm,xshift=1mm]frame.north east)
            arc[start angle=180,end angle=0,radius=1mm];
            \path[left color=tcbcolback!60!black,right color=tcbcolback!60!black,
                middle color=tcbcolback!80!black]
            ([xshift=-2mm]frame.north west) -- ([xshift=2mm]frame.north east)
            [rounded corners=1mm]-- ([xshift=1mm,yshift=-1mm]frame.north east)
            -- (frame.south east) -- (frame.south west)
            -- ([xshift=-1mm,yshift=-1mm]frame.north west)
            [sharp corners]-- cycle;
        }, interior engine=empty,
    },
    fonttitle=\bfseries,
    title={Bibliografía}, #1
}

% ==============================================================================
% ENTORNOS TEOREMA-LIKE CON DISEÑO MEJORADO
% ==============================================================================
\makeatletter
\newtcbtheorem{inciso}{Inciso}{             % Entorno tipo teorema
    enhanced,                               % Efectos visuales mejorados
    breakable,                              % Permite división entre páginas
    colback=white,                          % Fondo blanco
    colframe=azulpuma,                          % Color del marco (personalizable)
    attach boxed title to top left={         % Posición del título
        yshift*=-\tcboxedtitleheight
    },
    fonttitle=\bfseries,                    % Estilo de fuente del título
    title={#2},                             % Parámetro para título dinámico
    boxed title size=title,                 % Tamaño adaptado al contenido
    boxed title style={                     % Estilo del marco del título
        sharp corners,                      % Esquinas rectas
        rounded corners=northwest,          % Esquina noroeste redondeada
        colback=tcbcolframe,                % Color de fondo del título
        boxrule=0pt,                        % Sin borde adicional
    },
    underlay boxed title={                  % Efecto decorativo bajo el título
        \path[fill=tcbcolframe] (title.south west)--(title.south east)
        to[out=0, in=180] ([xshift=5mm]title.east)--
        (title.center-|frame.east)
        [rounded corners=\kvtcb@arc] |-
        (frame.north) -| cycle;
    },
    #1                                      % Parámetros adicionales
}{def}                                     % Etiqueta para referencias
\makeatother

% ==============================================================================
% CAJA DE NOTAS
% ==============================================================================
\usetikzlibrary{arrows,calc,shadows.blur}   % Requisitos para efectos gráficos
\tcbuselibrary{skins}                       % Habilita skins avanzados

\newtcolorbox{notas}[1][]{%
    enhanced jigsaw,                        % Skin mejorado para efectos
    colback=gray!20!white,                  % Fondo gris claro
    colframe=gray!80!black,                 % Borde gris oscuro
    size=small,                             % Tamaño compacto
    boxrule=1pt,                            % Grosor del borde
    title=\textbf{Notas | Observaciones},   % Título doble
    halign title=flush center,              % Alineación central del título
    coltitle=black,                         % Color del texto del título
    breakable,                              % Permite dividir en páginas
    drop shadow=black!50!white,             % Sombra exterior
    attach boxed title to top left={         % Posición del título
        xshift=1cm,
        yshift=-\tcboxedtitleheight/2,
        yshifttext=-\tcboxedtitleheight/2
    },
    minipage boxed title=5cm,               % Ancho máximo del título
    boxed title style={                     % Estilo personalizado del título
        colback=white,                      % Fondo blanco
        size=fbox,                         % Tamaño tipo marco
        boxrule=1pt,                        % Grosor del borde
        boxsep=2pt,                         % Espaciado interno
        underlay={                          % Elementos gráficos bajo el título
            \coordinate (dotA) at ($(interior.west) + (-0.5pt,0)$);
            \coordinate (dotB) at ($(interior.east) + (0.5pt,0)$);
            \begin{scope}
                \clip (interior.north west) rectangle ([xshift=3ex]interior.east);
                \filldraw [white, blur shadow={shadow opacity=60, shadow yshift=-.75ex}, 
                          rounded corners=2pt] (interior.north west) rectangle (interior.south east);
            \end{scope}
            \begin{scope}[gray!80!black]
                \fill (dotA) circle (2pt);  % Punto decorativo izquierdo
                \fill (dotB) circle (2pt);  % Punto decorativo derecho
            \end{scope}
        },
    },
    #1                                      % Personalización adicional
}

%%%%%%%%%%%%%%%%%%%%%%%%%%%%
% CONFIGURACIONES
%%%%%%%%%%%%%%%%%%%%%%%%%%%%

% ==============================================================================
%% Ajustes de Espaciado y Formato de Títulos
% ==============================================================================
% Ajusta el espaciado antes y después de los títulos de capítulos y secciones.
\titlespacing*{\chapter}{0cm}{-2.0cm}{0.50cm} % {izquierda}{espacio antes}{espacio después}
\titlespacing*{\section}{0cm}{0.50cm}{0.25cm}

% Configura la indentación y separación entre párrafos.
\setlength{\parindent}{0pt}  % Sin indentación en los párrafos
\setlength{\parskip}{1ex}    % Espacio extra entre párrafos

% ==============================================================================
%% Definición de Entorno Personalizado para Demostraciones
% ==============================================================================
\newtcbtheorem[]{demostracion}{Demostracion}%
{ enhanced,
  colback       = black!5,               % Color de fondo del recuadro
  colbacktitle  = black!5,               % Color de fondo del título
  coltitle      = black,                  % Color del texto del título
  boxrule       = 0pt,                    % Ancho del borde del recuadro
  frame hidden,                           % Oculta el marco del recuadro
  borderline west = {0.5mm}{0.0mm}{black},% Línea decorativa en el lateral izquierdo
  fonttitle     = \bfseries\sffamily,      % Fuente y estilo del título
  breakable,                              % Permite que el recuadro se divida en varias páginas
  before skip   = 3ex,                    % Espacio antes del entorno
  after skip    = 3ex                     % Espacio después del entorno
}{problem}

% ==============================================================================
%% Carga de Paquetes y Configuración de Matemáticas y Teoremas
% ==============================================================================
% Paquetes para símbolos, fuentes y entornos matemáticos
\usepackage{amssymb, latexsym, amsmath, amsthm}
\usepackage{amsfonts, rawfonts}   % Fuentes matemáticas adicionales
\usepackage{thmtools}             % Herramientas para mejorar la gestión de teoremas
\usepackage{fontawesome}           % Permite usar íconos y emojis sencillos

% Redefine la numeración de listas para usar letras minúsculas.
\renewcommand{\theenumi}{\alph{enumi})}

% Comandos personalizados para mejorar la notación matemática.
\let\emptyset\varnothing         % Utiliza el símbolo \varnothing para el conjunto vacío
\newcommand{\lto}{\mathbin{\to}}   % Define un comando para la flecha "→"

% Reestablece la indentación a 1cm y carga librerías adicionales para tcolorbox.
\setlength{\parindent}{1cm}
\tcbuselibrary{skins, breakable}

% Definición de entornos para teoremas, corolarios, definiciones y proposiciones.
\newtheorem{teorema}{Teorema}
\newtheorem{corolario}{Corolario}
\newtheorem{definición}{Definición}
\newtheorem{proposición}{Proposición}

% ==============================================================================
%% Configuración del Estilo del Documento y Encabezados
% ==============================================================================
\usepackage{epsfig, graphicx} % Inclusión de imágenes y gráficos
\usepackage[left=2cm, right=2cm, top=1.8cm, bottom=2.3cm]{geometry} % Define los márgenes del documento
\usepackage{fancyhdr}         % Para personalizar encabezados y pies de página
\usepackage{lastpage}         % Permite referenciar el número total de páginas

% Configuración de encabezados y pies con fancyhdr.
\pagestyle{fancy}
\fancyhf{}  % Limpia los encabezados y pies predeterminados
\rfoot{%
  \includegraphics[height=1cm]{Logo_FC.png} \hfill % Inserta un logo a la derecha
  Nombre del Estudiante | Número de cuenta | \textit{Página \thepage\ de \pageref{LastPage}}
}

% ==============================================================================
%% Paquetes Adicionales para Matemáticas y Utilidades
% ==============================================================================
\usepackage{amsmath} % Reafirma la carga de amsmath
\usepackage{amssymb, latexsym, amsmath, amsthm} % Carga de símbolos matemáticos y teoremas (reiterado para compatibilidad)
\usepackage{amssymb}
\usepackage{amsthm}
\usepackage[spanish]{babel}    % Configura el idioma del documento a español
\usepackage{geometry}         % Ya cargado anteriormente, se reafirma su uso
\usepackage{makecell}         % Para personalizar celdas en tablas
\usepackage{xcolor}           % Permite el uso de colores en el texto
\usepackage{amsfonts}
\usepackage{mathtools}        % Extiende funcionalidades de amsmath
\usepackage{tabularx, booktabs, ragged2e} % Herramientas para crear tablas de calidad
\usepackage{multirow}         % Permite celdas que abarcan varias filas en tablas
\usepackage{cancel}           % Para tachar términos en expresiones matemáticas
\usepackage{multicol}         % Permite el uso de múltiples columnas en el documento

% ==============================================================================
%% Texto de Relleno (Dummy Text)
% ==============================================================================
\usepackage{lipsum}           % Genera texto aleatorio para pruebas

% ==============================================================================
%% Herramientas para Gráficos y Diagramas
% ==============================================================================
\usetikzlibrary{arrows, backgrounds, shapes.arrows, positioning} % Librerías de TikZ para gráficos vectoriales
\usepackage{pgfplots, mathtools}
\pgfplotsset{compat=newest}   % Asegura la compatibilidad con la versión más reciente de pgfplots

% Redefine los operadores de parte real e imaginaria para evitar conflictos.
\let\Re\relax
\DeclareMathOperator{\Re}{Re}
\let\Im\relax
\DeclareMathOperator{\Im}{Im}
